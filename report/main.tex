%%%%%%%%%%%%%%%%%%%%%%%%%%%%%%%%%%%%%%%%%
% FRI Data Science_report LaTeX Template
% Version 1.0 (28/1/2020)
% 
% Jure Demšar (jure.demsar@fri.uni-lj.si)
%
% Based on MicromouseSymp article template by:
% Mathias Legrand (legrand.mathias@gmail.com) 
% With extensive modifications by:
% Antonio Valente (antonio.luis.valente@gmail.com)
%
% License:
% CC BY-NC-SA 3.0 (http://creativecommons.org/licenses/by-nc-sa/3.0/)
%
%%%%%%%%%%%%%%%%%%%%%%%%%%%%%%%%%%%%%%%%%


%----------------------------------------------------------------------------------------
%	PACKAGES AND OTHER DOCUMENT CONFIGURATIONS
%----------------------------------------------------------------------------------------
\documentclass[fleqn,moreauthors,10pt]{ds_report}
\usepackage[english]{babel}
\usepackage{minted}

\setminted{
  fontsize=\small, 
  breaklines=true, 
  breakafter={.,\space}, 
  breakbytokenanywhere=true, 
  frame=single,
}

\graphicspath{{fig/}}




%----------------------------------------------------------------------------------------
%	ARTICLE INFORMATION
%----------------------------------------------------------------------------------------

% Header
\JournalInfo{FRI Natural language processing course 2025}

% Interim or final report
\Archive{Project report} 
%\Archive{Final report} 

% Article title
\PaperTitle{Automatic generation of Slovenian traffic news for RTV Slovenija} 

% Authors (student competitors) and their info
\Authors{Anže Hočevar in Jan Anžur}

% Advisors
\affiliation{\textit{Advisors: Slavko Žitnik}}

% Keywords
\Keywords{Large language model, fine-tuning, data processing}
\newcommand{\keywordname}{Keywords}


%----------------------------------------------------------------------------------------
%	ABSTRACT
%----------------------------------------------------------------------------------------

\Abstract{
This project investigates the automatic generation of Slovenian traffic news using large language models (LLMs), aiming to emulate the structure and content style of RTV Slovenija bulletins. We propose and compare three data preparation pipelines—basic, DP1, and DP2—which align structured traffic events from \texttt{promet.si} with reference summaries from RTF reports. The GaMS-27B-Instruct model is fine-tuned using these pipelines via instruction-tuning and LoRA adaptation. Evaluation includes lexical overlap, edit distance, and semantic similarity metrics, alongside API-based LLM scoring. Results indicate that while DP1 yields the most structurally faithful outputs, basic and DP2 variants are often preferred by LLM evaluators due to their greater flexibility. The findings suggest a trade-off between rigid formatting and content adaptability, with implications for both training strategy and prompt design in multilingual, domain-specific text generation.
}

%----------------------------------------------------------------------------------------

\begin{document}

% Makes all text pages the same height
\flushbottom 

% Print the title and abstract box
\maketitle 

% Removes page numbering from the first page
\thispagestyle{empty} 

%----------------------------------------------------------------------------------------
%	ARTICLE CONTENTS
%----------------------------------------------------------------------------------------


	% These latex files are intended to serve as a the template for the NLP course at FRI.  The template is adapted from the FRI Data Science Project Competition. template  If you find mistakes in the template or have problems using it, please consult Jure Demšar (\href{mailto:jure.demsar@fri.uni-lj.si}{jure.demsar@fri.uni-lj.si}).
	
	% In the Introduction section you should write about the relevance of your work (what is the purpose of the project, what will we solve) and about related work (what solutions for the problem already exist). Where appropriate, reference scientific work conducted by other researchers. For example, the work done by Demšar et al. \cite{Demsar2016BalancedMixture} is very important for our project. The abbreviation et al. is for et alia, which in latin means and others, we use this abbreviation when there are more than two authors of the work we are citing. If there are two authors (or if there is a single author) we just write down their surnames. For example, the work done by Demšar and Lebar Bajec \cite{Demsar2017LinguisticEvolution} is also important for successful completion of our project.

\section*{Introduction}

Automated news generation (often called automated journalism or robot journalism) refers to software producing news from data with minimal human input \cite{acl_anthology}.
Early NLG systems in newsrooms were largely template-based, using predefined phrases filled with data (e.g., finance or sports reports) \cite{acl_anthology}.
Recent advances in deep learning and especially the Transformer architecture have greatly improved the ability to generate coherent, contextually accurate narratives from structured data \cite{arxiv_transformer}.
For example, Leppänen et al. (2017) developed a data-driven NLG system that produced thousands of localized election news articles in multiple languages \cite{acl_anthology}.
These systems highlight the potential of automated content creation, but also underscore requirements like transparency, accuracy, and adaptability in journalistic contexts \cite{acl_anthology}.
Ensuring factual correctness and neutrality is paramount – research notes that maintaining objectivity and avoiding bias remain significant challenges for AI-generated news \cite{arxiv_bias}.
Modern LLMs, with their ability to generate fluent text, are now being explored as core engines for such NLG tasks.

This dataset in question for this paper are short traffic news reports.
It includes periodically sampled data from the website of the Slovenian national motorway management company and reports that are broadcasted on a public radio station.
The objective is to enhance a large language model's (LLM's) ability to generate reports that align seamlessly with the style and structure of the provided input data by utilizing the appropriate natural processing techniques.



% or a part of intro?
\section*{Literature Review} 
One of the key considerations in automating Slovenian traffic news is the availability and adaptability of LLMs for multilingual text generation. Studies have shown that most state-of-the-art models, including OpenAI’s GPT series \cite{brown2020language}, BLOOM \cite{scao2022bloom}, and mT5 \cite{xue2021mt5}, demonstrate strong multilingual capabilities. However, Slovenian, being a low-resource language, remains under-represented in large-scale training corpora \cite{ulcar2021sloberta}. Locally fine-tuned models such as SloT5 \cite{ulcar2022slot5} have emerged to address this gap, showing promise in domain-specific Slovenian text generation.

A crucial challenge in this domain is balancing between **fine-tuning** and **prompt engineering**. Fine-tuning LLMs on domain-specific text can improve accuracy but requires computational resources and well-annotated data \cite{zhang2022fine}. Prompt engineering, on the other hand, provides a lighter-weight alternative by designing effective input prompts to guide the model’s response \cite{reynolds2021prompt}. For structured and time-sensitive content like traffic news, a hybrid approach may be necessary, combining a base model with well-optimized prompting techniques.

Parthasarathy et al. propose and describe a seven stage pipeline for fine-tuning an LLM that covers everything from data preparation to model evaluation \cite{ultimate2024}.
It also provides an overview of (parameter-efficient) fine-tuning techniques, enabling the selection of the optimal approach based on specific performance objectives and time efficiency requirements.

Additionally, context-aware traffic reporting can benefit from external data sources, such as live weather updates, public holiday schedules, and road congestion analytics.
Studies have indicated that integrating real-time sensor data and probabilistic event modelling improves the predictive accuracy of automated reports \cite{schelter2019automated}.
One option to explore is to include such data in the prompt at runtime, resembling a retrieval-augmented generation (RAG) chain approach, only with live data instead of extensive documents.

Based on these insights, the following sections analyse the provided structured dataset of Slovenian traffic news, examining patterns in report generation and urgency to inform an AI-based automation approach.




%------------------------------------------------

\section*{Methods}

\subsection*{Data}
We are working with traffic news data.
More specifically, we have a set of traffic news reports that serve as a part of our training data.
The other part is hidden in the periodically scraped website data that in some way or another corresponds to parts of the traffic news reports.
The input to our NLP model will be unseen scraped website data, whereas the output will be a report that follows the intended format.


\subsubsection*{Models}
We used an open-source large language model (LLM) named GaMS-27B-Instruct\footnote{https://huggingface.co/cjvt/GaMS-27B-Instruct} to generate new data.
Since our data is basically completely in Slovenian, we needed an LLM that was trained specifically in this language.
Technically speaking, it is a fine-tuned version of the Google Gemma2 LLM\footnote{https://huggingface.co/google/gemma-2-27b}, so it performs well both on English and Slovenian.

\subsubsection*{Data preprocessing variants}
We chose two implement two different preprocessing pipelines:
\begin{itemize}
    \item - Data preprocessing 1 (DP1), which creates input-output pairs of reports by taking an output (given a date) and pairing it with a flattened-input using multiple temporally-close inputs
    \item - Data preprocessing 2 (DP2), which splits the reports into paragraphs and tries to make input-output pairs using various NLP techniques for matching.
\end{itemize}

Both of these aim to take the given data and convert it into a specific format: a list of input-output pairs that we would like the LLM to ``learn from''.

\subsubsection*{Preprocessing Pipeline (DP1)}

To align structured traffic logs from \texttt{promet.si} with RTV Slovenija's RTF reports, we designed a multi-step preprocessing pipeline:

\paragraph{1. Temporal Filtering.}
We extract all structured events within a 1–8 hour window prior to the RTF timestamp, ensuring coverage of both new and persistent incidents.

\paragraph{2. Cleaning and Normalization.}
All fields are cleaned using HTML stripping, whitespace collapsing, and timestamp normalization to \texttt{datetime}.

\paragraph{3. Sentence Extraction and Deduplication.}
Each row is split into sentences and deduplicated using:
\begin{itemize}
  \item \textbf{Exact match:} Removes literal duplicates.
  \item \textbf{Semantic match:} Uses Sentence-BERT with cosine similarity $> 0.7$.
\end{itemize}

\paragraph{4. Content Selection.}
We retain key sentences via:
\begin{itemize}
  \item Longest informative sentence,
  \item TF-IDF scoring,
  \item Named entity filtering (e.g., roads, locations).
\end{itemize}

\paragraph{5. Input Formatting.}
Selected sentences are flattened into either:
\begin{itemize}
  \item[\textbf{(a)}] Plain concatenation (minimal preprocessing), or
  \item[\textbf{(b)}] RTV-style header with “\texttt{Prometne informacije DD.MM.YYYY HH.MM}” and “\texttt{Podatki o prometu.}” prefix.
\end{itemize}

\paragraph{6. RTF Matching.}
The closest-in-time RTF is parsed via regex to extract the reference summary.

This pipeline produces coherent model inputs structurally aligned with real RTV outputs and supports both evaluation and fine-tuning.


% \subsection*{Fine-tuning}
% In order for our data to be used in fine-tuning. we made input-output pairs from year 2022, as the procedure was long and computationally slow as well as keeping some information hidden from the model so it could be used as an evaluation set. 

\subsubsection*{Preprocessing pipeline (DP2)}
This pipeline is somewhat simpler in nature.
It takes a given output and finds its best possible corresponding input.
The stages include:
\begin{itemize}
    \item Split given report (output) into individual events.
    \item Find and split temporally-close input data (from excel).
    % \item Split input data also into events.
    \item Compare each event from the given report with all events from the input.
    \item Create input-output pairs using matches that meet a confidence threshold.
\end{itemize}

\paragraph{1. Splitting into events.}
After visually observing the dataset, we observed that the outputs (individual RTF files) are split into multiple paragraphs and that each paragraph corresponds to one event or a group of related events.
In other words, the contents of different paragraphs pertain to different news stories.
Additionally, each line in the input data (excel line) consist of multiple columns that can be (carefully) merged.
After being merged, we notice that this data can also be split into events by the html tags.
After this stage, we split every data point in the input and output into a series of events.

\paragraph{2. Finding temporally-close input data.}
This stage is fairly similar to its analogue from DP1.
We simply take all the input data from a time-window around the time that the output data was obtained and split it into a flat list of events (outputs).

\paragraph{3. Matching inputs to a given output.}
After the previous stage, we have a single output event and a list of candidate input events.
We implemented a multi-stage paragraph-comparison procedure.
This procedure takes into account the number of matching words, proper nouns (PN) and named entities (NE).
The latter two have a significant overlap since they both roughly refer to a specific person, location, object etc., but are not the same in their implementation.
We searched for pairs of matching words using Levenshtein distance on the lemmatized paragraphs (maximum distance of 1 for a match).
Any given input-output pair (of events) had to pass a series of trials, where it would need to pass an adjusted minimum threshold for the number matching words, PNs and NEs.
Finally, the pair needed to also achieve a high enough score using vector embeddings comparison.



\section*{Baseline results}

% gams-27B-Instruct

As a starting point, we attempted to approach this task by simply using just common prompt engineering methods.
To put it more bluntly, we fed the LLM with a prompt that concisely and clearly describes the task at hand and provided a sequence of shots (examples) that it should be able to follow.
An example of this approach in action is shown in figure~\ref{fig:result_baseline}.
It is apparent that the output is very similar to the input.
Although, this is not necessarily bad, as long as the rules for data generation are not violated.


\section*{Fine-Tuning: Training Data Generation and Model Adaptation}
To fine-tune our large language models (LLMs) on domain-specific Slovenian traffic news reports, we prepared a high-quality set of input–output training examples derived from publicly available sources. This pipeline is computationally intensive due to large-scale RTF parsing, semantic deduplication, and format conversion.
% As such, we preprocessed and extracted all pairs for the year 2022, leaving 2023–2024 for evaluation and held-out testing.

\paragraph{1. Input–Output Pair Generation.}
These pairs are stored in JSONL format for training with one object per line:
\{ "input": "...", "output": "..." \}.
We convert structured excel logs and corresponding RTF summaries into training examples in the format:

DP1:

VHOD:
\textless flattened, cleaned input paragraph \textgreater

IZHOD:
\textless RTF-sourced summary matching the input's timestamp context (ground truth) \textgreater
\\

DP2:

VHOD:
\textless the input part of a match \textgreater

IZHOD:
\textless the output part of the match \textgreater

\paragraph{2. Prompt Construction.}
We then formatted these pairs into an instruction-tuning template. It includes:

A task description (e.g. “You are a traffic report writer...”),

Three few-shot examples,

The current input as the final prompt, expecting model completion.

The style would be similar to what the LLM would recieve as input for inference after being fine-tuned.

\subsection*{3. Training Setup.}
Models such as GaMS-2B, GaMS-9B-Instruct and GaMS-27B are fine-tuned using PEFT (LoRA) on each of the data preparation techniques, leveraging Hugging Face pipelines and mixed-precision execution. Inference outputs are saved in outputs.jsonl for evaluation. 

\textbf{LoRA Configuration.} The model was fine-tuned using Low-Rank Adaptation (LoRA) with the following parameters:

\begin{itemize}
  \item \textbf{\texttt{task\_type} = \texttt{CAUSAL\_LM}} — Targets causal language modeling tasks.
  \item \textbf{\texttt{r} = 64} — Rank of the low-rank adapter matrices.
  \item \textbf{\texttt{lora\_alpha} = 64} — Scaling factor applied to the adapter outputs.
  \item \textbf{\texttt{lora\_dropout} = 0.1} — Dropout probability applied to adapter layers.
  \item \textbf{\texttt{bias} = \texttt{"none"}} — Bias terms are not adapted.
\end{itemize}

\section*{Evaluation}

To assess the quality of model-generated outputs, we conducted inference across three distinct data preparation pipelines: \textit{Basic}, \textit{DP1}, and \textit{DP2}. Each setting was evaluated over 20 randomly selected samples.

The \textbf{Basic prompt} consists of raw traffic event descriptions concatenated with minimal preprocessing and embedded HTML tags. An example of the generated output under this configuration is shown in Figure~\ref{fig:result_baseline}. As a simple enhancement, we then applied basic text cleaning—stripping HTML tags and collapsing whitespace—which resulted in improved coherence and stylistic alignment with human-written summaries. This cleaned variant is illustrated in Figure~\ref{fig:result_basic}.

To further increase relevance and faithfulness to human reports, we developed two structured pipelines:
\begin{itemize}
    \item \textbf{DP1} leverages temporal filtering of structured data based on timestamp alignment with a single target RTF report. The result is a longer, context-rich prompt paired with a verified human-written summary, enabling direct output–reference comparison (Figure~\ref{fig:result_dp1}).
    
    \item \textbf{DP2}, in contrast, doesn't necessarily take the whole output RTF into account.
    It matches paragraphs from it to paragraphs from a flattened list of inputs.
    While this makes precise reference-based evaluation more difficult, the generated output (Figure~\ref{fig:result_dp2}) remains interpretable and suitable for qualitative inspection.
\end{itemize}

These examples, provided in the appendix, highlight the model’s varying performance across input strategies and offer insight into the impact of preprocessing on generation quality.


\begin{table}[ht]
  \centering
  \caption{Summary of evaluation metrics on generated vs.\ ground-truth traffic reports}
  \label{tab:metrics_summary}
  \begin{tabular}{lrr}
    \toprule
    \textbf{Metric}           & \textbf{Median} & \textbf{Std.\ Dev.} \\
    \midrule
    F1 Token Overlap          & 0.3877          & 0.0553              \\
    Jaccard Similarity        & 0.2405          & 0.0433              \\
    BLEU Score                & 0.0898          & 0.0432              \\
    ROUGE-L F\textsubscript{measure} & 0.2859    & 0.0474              \\
    Levenshtein Ratio         & 0.5118          & 0.0389              \\
    Embedding Similarity      & 0.8620          & 0.0399              \\
    Precision (Tokens)        & 0.3491          & 0.0760              \\
    Recall (Tokens)           & 0.4208          & 0.1100              \\
    \bottomrule
  \end{tabular}
\end{table}


The suite of metrics captures three facets of quality:
\begin{enumerate}
  \item \textbf{Lexical Overlap:}
    F1 Token Overlap (\(\approx 0.39\)) and Jaccard Similarity (\(\approx 0.24\)) indicate moderate word‐level agreement.
    BLEU (\(\approx 0.09\)) and ROUGE-L (\(\approx 0.29\)) remain low, as they penalize missing or reordered n-grams in these relatively long texts.
  \item \textbf{Character-Level Edit Distance:}
    The Levenshtein Ratio (\(\approx 0.51\)) reveals that only about half of all characters align in sequence, reflecting substantial paraphrasing.
  \item \textbf{Semantic Fidelity:}
    A high Embedding Similarity (\(\approx 0.86\)) demonstrates that, despite surface‐form differences, the model’s outputs convey nearly the same meaning as the ground truth.
    Token-level Precision (\(\approx 0.35\)) vs.\ Recall (\(\approx 0.42\)) suggests the model includes extra or varied content (lower precision) while covering most ground-truth concepts (higher recall).
\end{enumerate}

\noindent
\textbf{Conclusion:}
Although literal n-gram overlap is limited — resulting in low BLEU/ROUGE scores—the strong semantic correspondence suggests the core content is well captured. Future work should target entity-level fidelity and surface-form consistency (e.g., via constrained decoding or slot-filling) to boost lexical metrics without sacrificing meaning.


\subsection*{API LLM evaluation}

As shown in Table~\ref{tab:evaluation-summary}, \emph{basic\_outputs.jsonl} scored highest (4.05), \emph{dp2\_outputs.jsonl} was intermediate (3.91), and \emph{dp1\_outputs.jsonl} scored lowest (2.85). Scores were obtained via an API-driven LLM (DeepSeek V3 0324) prompted to rate each bulletin (1--10) on structure and content against RTF exemplars. This contrasts with earlier automated and structural analyses that favored the \emph{dp1} format, suggesting human judges prize flexibility and contextual clarity over rigid templating. The large standard deviations further reveal variability in bulletin quality.

Interestingly, both fine-tuned variants achieved a somewhat lower score than the basic variant.
This puts into question the effectiveness of the FT pipeline.
However, finding the main culprit for this result is not very straightforward.
Both data preprocessing and FT procedure include many parameters that can be futher tweaked.
Additionally, the quantity, quality and variety of the data itself surely has some impact on the results.

\begin{table}[ht]
  \centering
  \caption{DeepSeek V3 API Evaluation Scores Summary}
  \label{tab:evaluation-summary}
  \begin{tabular}{lcc}
    \toprule
    Dataset               & Average Score & Std.\ Dev.\ \\
    \midrule
    basic\_outputs.jsonl  & 4.05          & 1.9049      \\
    dp1\_outputs.jsonl    & 2.85          & 1.1522      \\
    dp2\_outputs.jsonl    & 3.9130        & 1.8630      \\
    \bottomrule
  \end{tabular}
\end{table}






% Use the Methods section to describe what you did an how you did it -- in what way did you prepare the data, what algorithms did you use, how did you test various solutions ... Provide all the required details for a reproduction of your work.

% Below are \LaTeX examples of some common elements that you will probably need when writing your report (e.g. figures, equations, lists, code examples ...).


% \subsection*{Equations}

% You can write equations inline, e.g. $\cos\pi=-1$, $E = m \cdot c^2$ and $\alpha$, or you can include them as separate objects. The Bayes’s rule is stated mathematically as:

% \begin{equation}
% 	P(A|B) = \frac{P(B|A)P(A)}{P(B)},
% 	\label{eq:bayes}
% \end{equation}

% where $A$ and $B$ are some events. You can also reference it -- the equation \ref{eq:bayes} describes the Bayes's rule.

% \subsection*{Lists}

% We can insert numbered and bullet lists:

% % the [noitemsep] option makes the list more compact
% \begin{enumerate}[noitemsep] 
% 	\item First item in the list.
% 	\item Second item in the list.
% 	\item Third item in the list.
% \end{enumerate}

% \begin{itemize}[noitemsep] 
% 	\item First item in the list.
% 	\item Second item in the list.
% 	\item Third item in the list.
% \end{itemize}

% We can use the description environment to define or describe key terms and phrases.

% \begin{description}
% 	\item[Word] What is a word?.
% 	\item[Concept] What is a concept?
% 	\item[Idea] What is an idea?
% \end{description}


% \subsection*{Random text}

% This text is inserted only to make this template look more like a proper report. Lorem ipsum dolor sit amet, consectetur adipiscing elit. Etiam blandit dictum facilisis. Lorem ipsum dolor sit amet, consectetur adipiscing elit. Interdum et malesuada fames ac ante ipsum primis in faucibus. Etiam convallis tellus velit, quis ornare ipsum aliquam id. Maecenas tempus mauris sit amet libero elementum eleifend. Nulla nunc orci, consectetur non consequat ac, consequat non nisl. Aenean vitae dui nec ex fringilla malesuada. Proin elit libero, faucibus eget neque quis, condimentum laoreet urna. Etiam at nunc quis felis pulvinar dignissim. Phasellus turpis turpis, vestibulum eget imperdiet in, molestie eget neque. Curabitur quis ante sed nunc varius dictum non quis nisl. Donec nec lobortis velit. Ut cursus, libero efficitur dictum imperdiet, odio mi fermentum dui, id vulputate metus velit sit amet risus. Nulla vel volutpat elit. Mauris ex erat, pulvinar ac accumsan sit amet, ultrices sit amet turpis.

% Phasellus in ligula nunc. Vivamus sem lorem, malesuada sed pretium quis, varius convallis lectus. Quisque in risus nec lectus lobortis gravida non a sem. Quisque et vestibulum sem, vel mollis dolor. Nullam ante ex, scelerisque ac efficitur vel, rhoncus quis lectus. Pellentesque scelerisque efficitur purus in faucibus. Maecenas vestibulum vulputate nisl sed vestibulum. Nullam varius turpis in hendrerit posuere.


% \subsection*{Figures}

% You can insert figures that span over the whole page, or over just a single column. The first one, \figurename~\ref{fig:column}, is an example of a figure that spans only across one of the two columns in the report.

% \begin{figure}[ht]\centering
% 	\includegraphics[width=\linewidth]{single_column.pdf}
% 	\caption{\textbf{A random visualization.} This is an example of a figure that spans only across one of the two columns.}
% 	\label{fig:column}
% \end{figure}

% On the other hand, \figurename~\ref{fig:whole} is an example of a figure that spans across the whole page (across both columns) of the report.

% % \begin{figure*} makes the figure take up the entire width of the page
% \begin{figure*}[ht]\centering 
% 	\includegraphics[width=\linewidth]{whole_page.pdf}
% 	\caption{\textbf{Visualization of a Bayesian hierarchical model.} This is an example of a figure that spans the whole width of the report.}
% 	\label{fig:whole}
% \end{figure*}


% \subsection*{Tables}

% Use the table environment to insert tables.

% \begin{table}[hbt]
% 	\caption{Table of grades.}
% 	\centering
% 	\begin{tabular}{l l | r}
% 		\toprule
% 		\multicolumn{2}{c}{Name} \\
% 		\cmidrule(r){1-2}
% 		First name & Last Name & Grade \\
% 		\midrule
% 		John & Doe & $7.5$ \\
% 		Jane & Doe & $10$ \\
% 		Mike & Smith & $8$ \\
% 		\bottomrule
% 	\end{tabular}
% 	\label{tab:label}
% \end{table}


% \subsection*{Code examples}

% You can also insert short code examples. You can specify them manually, or insert a whole file with code. Please avoid inserting long code snippets, advisors will have access to your repositories and can take a look at your code there. If necessary, you can use this technique to insert code (or pseudo code) of short algorithms that are crucial for the understanding of the manuscript.

% \lstset{language=Python}
% \lstset{caption={Insert code directly from a file.}}
% \lstset{label={lst:code_file}}
% \lstinputlisting[language=Python]{code/example.py}

% \lstset{language=R}
% \lstset{caption={Write the code you want to insert.}}
% \lstset{label={lst:code_direct}}
% \begin{lstlisting}
% import(dplyr)
% import(ggplot)

% ggplot(diamonds,
% 	   aes(x=carat, y=price, color=cut)) +
%   geom_point() +
%   geom_smooth()
% \end{lstlisting}


% Use the results section to present the final results of your work. Present the results in a objective and scientific fashion. Use visualisations to convey your results in a clear and efficient manner. When comparing results between various techniques use appropriate statistical methodology.

% \subsection*{More random text}

% This text is inserted only to make this template look more like a proper report. Lorem ipsum dolor sit amet, consectetur adipiscing elit. Etiam blandit dictum facilisis. Lorem ipsum dolor sit amet, consectetur adipiscing elit. Interdum et malesuada fames ac ante ipsum primis in faucibus. Etiam convallis tellus velit, quis ornare ipsum aliquam id. Maecenas tempus mauris sit amet libero elementum eleifend. Nulla nunc orci, consectetur non consequat ac, consequat non nisl. Aenean vitae dui nec ex fringilla malesuada. Proin elit libero, faucibus eget neque quis, condimentum laoreet urna. Etiam at nunc quis felis pulvinar dignissim. Phasellus turpis turpis, vestibulum eget imperdiet in, molestie eget neque. Curabitur quis ante sed nunc varius dictum non quis nisl. Donec nec lobortis velit. Ut cursus, libero efficitur dictum imperdiet, odio mi fermentum dui, id vulputate metus velit sit amet risus. Nulla vel volutpat elit. Mauris ex erat, pulvinar ac accumsan sit amet, ultrices sit amet turpis.

% Phasellus in ligula nunc. Vivamus sem lorem, malesuada sed pretium quis, varius convallis lectus. Quisque in risus nec lectus lobortis gravida non a sem. Quisque et vestibulum sem, vel mollis dolor. Nullam ante ex, scelerisque ac efficitur vel, rhoncus quis lectus. Pellentesque scelerisque efficitur purus in faucibus. Maecenas vestibulum vulputate nisl sed vestibulum. Nullam varius turpis in hendrerit posuere.

% Nulla rhoncus tortor eget ipsum commodo lacinia sit amet eu urna. Cras maximus leo mauris, ac congue eros sollicitudin ac. Integer vel erat varius, scelerisque orci eu, tristique purus. Proin id leo quis ante pharetra suscipit et non magna. Morbi in volutpat erat. Vivamus sit amet libero eu lacus pulvinar pharetra sed at felis. Vivamus non nibh a orci viverra rhoncus sit amet ullamcorper sem. Ut nec tempor dui. Aliquam convallis vitae nisi ac volutpat. Nam accumsan, erat eget faucibus commodo, ligula dui cursus nisi, at laoreet odio augue id eros. Curabitur quis tellus eget nunc ornare auctor.


%------------------------------------------------

\section*{Discussion}

\subsection*{Qualitative Analysis of Preparation Strategies}
To better understand how data preparation impacts generation quality, we compared representative outputs from all three variants. DP1 yielded bulletins with high structural fidelity, closely mimicking RTV Slovenija formatting (e.g., \textit{road}~$\rightarrow$~\textit{direction}~$\rightarrow$~\textit{event}~$\rightarrow$~\textit{consequence}). However, its rigid template often caused repetitive phrasing and excluded minor events outside the 8-hour window.

DP2 outputs, while lacking formal headers, showed more fluent and human-like summaries, often merging adjacent incidents effectively. These captured contextual nuance better, especially when events spanned multiple RTF paragraphs.

The basic variant produced mixed results: broader inclusion improved recall but occasionally introduced outdated or misaligned events due to the absence of temporal filtering.

Overall, LLM-based evaluations (e.g., DeepSeek V3) favored DP2 and basic over DP1, highlighting a trade-off between structural alignment (DP1) and natural summarization (DP2/basic). Future strategies may benefit from blending structural templates with adaptive, entity-aware inputs.

% 

\subsection*{Future Work}

Future improvements could include integrating \emph{named-entity alignment metrics} to better assess factual consistency (e.g., road names, locations). Adding \emph{BERTScore} or other semantic-aware metrics would complement lexical evaluations.

On the data side, \emph{adaptive temporal filtering} and hybrid strategies that combine DP1's structure with DP2's flexibility could improve both coverage and style.

Model-wise, exploring \emph{constrained decoding} or \emph{reinforcement tuning} (e.g., RLHF) may help enforce formatting and factual accuracy. Finally, incorporating \emph{external data sources} such as weather or live traffic sensors could enable real-time applications.




%------------------------------------------------

% \section*{Acknowledgments}

% Here you can thank other persons (advisors, colleagues ...) that contributed to the successful completion of your project.


%----------------------------------------------------------------------------------------
%	REFERENCE LIST
%----------------------------------------------------------------------------------------
\bibliographystyle{unsrt}
\bibliography{report}

\clearpage
\begin{figure*}[t]
    \centering
    \caption*{\textbf{Appendix: Prompt–Response Format Example}}

    \begin{minted}[frame=single, fontsize=\small]{text}
    VHOD:
    <p><strong>Zastoji</strong></p>
    <p>- na ljubljanski južni obvoznici od Rudnika proti razcepu Kozarje;
        <br>- na gorenjski avtocesti od Kosez do predora Šentvid proti Kranju;
        <br>- na štajerski avtocesti med Blagovico in Trojanami proti Celju ter pred prehodom Šentilj proti Avstriji;
        <br>- na ljubljanski zahodni obvoznici od Kosez proti Kozarjam ter na primorski avtocesti med Ljubljano in Brezovico proti Kopru;
        <br>- na cestah Ljubljana - Brezovica, Lesce - Bled in Šmarje - Koper.
    </p>

    IZHOD:
    Zastoji so na cestah Šmarje-Koper in Lesce-Bled.
    Prav tako so zastoji na zahodni in južni ljubljanski obvoznici od Kosez in Rudnika proti Primorski. Potovalni čas se na obeh odsekih podaljša za približno 15 minut.
    Zaradi del je krajši zastoj na štajerski avtocesti med Blagovico in Trojanami proti Mariboru. Opozarjamo na nevarnost naleta.
    \end{minted}

    \caption{An example of LLM output generation for given input data using only prompt engineering methods. 
    The input (VHOD) is a flattened and HTML-tagged paragraph derived from structured traffic events, while the output (IZHOD) is a free-form textual summary in the style of RTV Slovenija traffic reports.}
    \label{fig:result_baseline}
\end{figure*}

% This example illustrates the format used during inference and instruction fine-tuning. The input (VHOD) is a flattened and HTML-tagged paragraph derived from structured traffic events, while the output (IZHOD) is a free-form textual summary in the style of RTV Slovenija traffic reports.

    % VHOD:
    % <p><strong>Zastoji</strong></p>
    % <p>- na ljubljanski južni obvoznici od Rudnika proti razcepu Kozarje;
    %     <br>- na gorenjski avtocesti od Kosez do predora Šentvid proti Kranju;
    %     <br>- na štajerski avtocesti med Blagovico in Trojanami proti Celju ter pred prehodom Šentilj proti Avstriji;
    %     <br>- na ljubljanski zahodni obvoznici od Kosez proti Kozarjam ter na primorski avtocesti med Ljubljano in Brezovico proti Kopru;
    %     <br>- na cestah Ljubljana - Brezovica, Lesce - Bled in Šmarje - Koper.
    % </p>
\begin{figure*}[t]
    \centering
    \caption*{\textbf{Appendix: Basic Prompt–Response Format Example}} % TITLE GOES HERE
    \vspace{1mm}
    \begin{minted}[frame=single, fontsize=\small]{text}
    VHOD:
    Na štajerski avtocesti med priključkoma Slovenska Bistrica sever in Fram promet poteka po dveh zoženih pasovih v obe smeri.
    Cesta Litija - Zagorje bo zaprta pri Šklendrovcu v soboto, 18. maja, med 7.30 in 17. uro.
    Več o delovnih zaporah v prometni napovedi.
    Oviran promet na gorenjski avtocesti med predorom Šentvid in Podutikom proti Kosezam.
    Regionalna cesta Sevnica - Planina pri Sevnici je zaprta pri Orešju nad Sevnico.
    Okvara vozila na primorski avtocesti pred razcepom Gabrk proti Kopru.
    Prireditve ta konec tedna.
    Zaradi poplavljenega vozišča je zaprta cesta Lešje - Koritno v občini Majšperk.
    Povečan promet na cestah iz smeri Avstrije proti Hrvaški. Zastoji:- na gorenjski avtocesti prei priključku Jesenice vzhod proti Ljubljani;- upočasnjen promet na primorski avtocesti od Kozarij proti Brezovici;- na glavni cesti Postojna - Pivka, med Matenjo vasjo in Prestrankom.


    IZHOD:

    Na štajerski avtocesti med Slovensko Bistrico in Framom promet poteka po dveh zoženih pasovih v obe smeri.
    Cesta Litija - Zagorje bo zaprta pri Šklendrovcu v soboto, 18. maja, med 7.30 in 17. uro.
    Okvara vozila ovira promet na primorski avtocesti pred razcepom Gabrk proti Kopru.
    Oviran promet na gorenjski avtocesti med predorom Šentvid in Podutikom proti Kosezam.
    Regionalna cesta Sevnica - Planina pri Sevnici je zaprta pri Orešju nad Sevnico.
    Na cestah iz smeri Avstrije proti Hrvaški je povečan prometni tok. Zastoji so na gorenjski avtocesti pred priključkom Jesenice vzhod proti Ljubljani, upočasnjen promet na primorski avtocesi med Kozarijami in Brezovico, ter na glavni cesti med Matenjo vasjo in Prestrankom.

    \end{minted}
    \caption{Basic example.}
    \label{fig:result_basic}
\end{figure*}


\begin{figure*}[t]
    \centering
    \caption*{\textbf{Appendix: DP1 Prompt–Response Format Example}}

    \begin{minted}[frame=single, fontsize=\small]{text}
    VHOD:

    Prometne informacije       18. 08. 2024   \t   01.04           2. program
    
    Podatki o prometu.
    
    Zastoji Gorenjska: - Na gorenjski avtocesti med Lescam in predorom Karavanke proti Avstriji, 10 km. Izvoz Lipce je dovoljen samo za lokalni promet. - Na cestah Jesenice - Hrušica in Podkoren - Korensko sedlo. Delo na cesti Cesta Senarska - Lenart , pri Sveti Trojici bo zaprta do nedelje,18. avgusta, do 20. ure. Obvoz bo potekal po državnih cestah in po avtocesti med priključkoma Lenart – Sv. Trojica in obratno, tudi za vozila brez vinjete vendar samo med tema dvema priključkoma. Več o delovnih zaporah v prometni napovedi . Buy vignette for Slovenia online Thererore long queues are expected in entering points from Austria to Slovenia, i.e. Karavanke tunnel (A2) and Sentilj/Spielfeld crossing (A1). Important reason for these queues is that drivers don't have vignette for Slovenian roads and have to buy them at the border. To reduce or even avoid long waiting periods drivers are strongly recommended to buy vignette for Slovenian motorways online. They can do it here . DARS Janko Poženel, GNC Prireditve  Tovorni promet   Italija
    
    IZHOD:
    Prometne informacije        18. 08. 2024       01.04           2. program
    
    Podatki o prometu.
    
    Na primorski avtocesti je zastoj na izvozu Brezovica proti Kopru, čas vožnje se tam podaljša za približno 10 minut.
    
    Na gorenjski avtocesti je zastoj pred predorom Karavanke proti Avstriji, čas potovanja se tam podaljša za približno pol ure.
    
    Na glavni cesti Maribor-Hoče so zastoji v obe smeri med priključkom Maribor-vzhod in krožiščem Pobrežje. V smeri proti Mariboru je promet oviran na vipavski hitri cesti med priključkoma Vipava in Selo, zaradi pokvarjenega vozila.
    
    Na mejnih prehodih Gruškovje in Obrežje vozniki tovornih vozil na vstop v državo čakajo 2 uri, na Obrežju pa tudi na izstop.


    GROUND TRUTH:
    Prometne informacije      18.08.2024              05.00          1. in 2. program
    
    Podatki o prometu.
    
    Na štajerski avtocesti je zaradi prometne nesreče pred priključkom Fram proti Ljubljani zaprt vozni pas.
    
    Na pomurski avtocesti je na izvozu Cerkvenjak proti Mariboru zaradi pokvarjenega vozila oviran promet.
    
    Na gorenjski avtocesti je med Lescam in predorom Karavanke proti Avstriji približno 2 kilometre dolg zastoj. Izvoz Lipce je dovoljen samo za lokalni promet.
    
    Taradi prireditve bo v Ljubljani od 8-ih do 15-ih zaprta Štajerska cesta, med rondojem Tomačevo in rondojem s Kranjčevo cesto.
    
    Cesta Senarska - Lenart pa bo zaradi del pri Sveti Trojici zaprta do 20-tih. Obvoz je po državnih cestah in po avtocesti med priključkoma Lenart - Sv. Trojica in obratno, tudi za vozila brez vinjete vendar samo med tema dvema priključkoma.

    \end{minted}

    \caption{Example of output from GaMS-27B-Instruct using the DP1 method for prompt construction and fine-tuning. The corresponding ground truth summary is shown for comparison.}
    \label{fig:result_dp1}
\end{figure*}

\begin{figure*}[t]
    \centering
    \caption*{\textbf{Appendix: DP2 Prompt–Response Format Example}}

    \begin{minted}[frame=single, fontsize=\small]{text}
    VHOD:

    Danes ponoči bo skozi predor Karavanke od 20. do 5. ure, promet potekal izmenično s čakanjem pred predorom predvidoma 30 minut.
    Na ljubljanski severni obvoznici potekajo dela med Zadobrovo in Tomačevim. Med Novimi Jaršami in Zadobrovo promet proti Zaloški poteka po dveh zoženih pasovih.
    Čakalna doba: Obrežje in Gruškovje.
    Na štajerski avtocesti med Polskavo in Framom proti Mariboru zaprt vozni pas. Nastaja zastoj, 1,8 kilometra.
    Zaprte ceste zaradi gašenja požara na Primorskem:- regionalna cesta Črni Kal - Gabrovica pri Črnem Kalu - Osp - meja z Italijo;- lokalna cesta med Črnim Kalom in Staro Gabrovico;- ter lokalna cesta na relaciji avtocestni priključek Kastelec - vas Kastelec - Socerb - meja z Italijo.
    Na cesti Bohinjska Bela - Bled, na Bledu.
    
    IZHOD:
    
    Na gorenjski avtocesti proti Ljubljani je zaradi nesreče zaprt prehitevalni pas med priključkoma Brezje in Podtabor. Nastal je zastoj, opozarjamo na nevarnost naleta. Proti Karavankam pa je zaradi pokvarjenega vozila zaprt vozni pas pred predorom Ljubno. Zaradi del bo na gorenjski avtocesti med 11-to in 16-to uro promet skozi predor Karavanke potekal izmenično enosmerno. Na mejnem prehodu Obrežje vozniki tovornih vozil na vstop v državo čakajo 2 uri, v Gruškovju pa do pol ure. Zaradi gašenja požara so zaprte regionalna cesta Črni Kal-Gabrovica pri Črnem Kalu-Osp-meja z Italijo, lokalna cesta Črni Kal-Stara Gabrovica ter lokalna cesta od avtocestnega priključka Kastelec, skozi Socerb, do meje z Italijo. Zastoj je na cesti Bohinjska Bela-Bled.


    \end{minted}

    \caption{Example of output from GaMS-27B-Instruct using the DP2 method for prompt construction and fine-tuning.}
    \label{fig:result_dp2}
\end{figure*}


\begin{figure*}[t]
    \centering
    \caption*{\textbf{Appendix: Manual ChatGPT Comparative Evaluation of Output Sets}}

    \begin{minted}[frame=single, fontsize=\small]{text}
Output Set        Score  
-------------------------
basic_outputs     4 / 10  
dp1_outputs       9 / 10  
dp2_outputs       6 / 10  

Rationale:
1. basic_outputs (4/10)
   - No standard header (“Prometne informacije ...”).
   - Inconsistent ordering of road, direction, event.
   - Missing “Podatki o prometu.” section title.
   - Uneven naming conventions and event grouping.

2. dp1_outputs (9/10)
   - Correct header and date/time/program line.
   - Includes “Podatki o prometu.” before events.
   - Consistent sentence structure:
       road + direction → event → consequence.
   - Proper motorway names and Slovene terminology.
   - Minor slip-ups in program numbering/order.

3. dp2_outputs (6/10)
   - Well-formed event sentences (location, reason, impact).
   - Omits the bulletin framing (header + title).
   - Lacks blank lines separating items.
   - Presents raw list rather than full RTF-style bulletin.
    \end{minted}

    \caption{Scores and qualitative rationale for each output set,  
             based on the formatting rules in \texttt{PROMET.docx}  
             and style exemplars in the provided RTF files by ChatGPT.}
    \label{fig:appendix_comparison}
\end{figure*}

\end{document}