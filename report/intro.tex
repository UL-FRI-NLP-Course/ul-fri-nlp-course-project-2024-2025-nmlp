\section*{Introduction}

Automated news generation (often called automated journalism or robot journalism) refers to software producing news from data with minimal human input \cite{acl_anthology}.
Early NLG systems in newsrooms were largely template-based, using predefined phrases filled with data (e.g., finance or sports reports) \cite{acl_anthology}.
Recent advances in deep learning and especially the Transformer architecture have greatly improved the ability to generate coherent, contextually accurate narratives from structured data \cite{arxiv_transformer}.
For example, Leppänen et al. (2017) developed a data-driven NLG system that produced thousands of localized election news articles in multiple languages \cite{acl_anthology}.
These systems highlight the potential of automated content creation, but also underscore requirements like transparency, accuracy, and adaptability in journalistic contexts \cite{acl_anthology}.
Ensuring factual correctness and neutrality is paramount – research notes that maintaining objectivity and avoiding bias remain significant challenges for AI-generated news \cite{arxiv_bias}.
Modern LLMs, with their ability to generate fluent text, are now being explored as core engines for such NLG tasks.
