\begin{figure*}[t]
    \centering
    \caption*{\textbf{Appendix: Prompt–Response Format Example}}

    \begin{minted}[frame=single, fontsize=\small]{text}
    VHOD:
    <p><strong>Zastoji</strong></p>
    <p>- na ljubljanski južni obvoznici od Rudnika proti razcepu Kozarje;
        <br>- na gorenjski avtocesti od Kosez do predora Šentvid proti Kranju;
        <br>- na štajerski avtocesti med Blagovico in Trojanami proti Celju ter pred prehodom Šentilj proti Avstriji;
        <br>- na ljubljanski zahodni obvoznici od Kosez proti Kozarjam ter na primorski avtocesti med Ljubljano in Brezovico proti Kopru;
        <br>- na cestah Ljubljana - Brezovica, Lesce - Bled in Šmarje - Koper.
    </p>

    IZHOD:
    Zastoji so na cestah Šmarje-Koper in Lesce-Bled.
    Prav tako so zastoji na zahodni in južni ljubljanski obvoznici od Kosez in Rudnika proti Primorski. Potovalni čas se na obeh odsekih podaljša za približno 15 minut.
    Zaradi del je krajši zastoj na štajerski avtocesti med Blagovico in Trojanami proti Mariboru. Opozarjamo na nevarnost naleta.
    \end{minted}

    \caption{An example of LLM output generation for given input data using only prompt engineering methods. 
    The input (VHOD) is a flattened and HTML-tagged paragraph derived from structured traffic events, while the output (IZHOD) is a free-form textual summary in the style of RTV Slovenija traffic reports.}
    \label{fig:result_baseline}
\end{figure*}

% This example illustrates the format used during inference and instruction fine-tuning. The input (VHOD) is a flattened and HTML-tagged paragraph derived from structured traffic events, while the output (IZHOD) is a free-form textual summary in the style of RTV Slovenija traffic reports.

    % VHOD:
    % <p><strong>Zastoji</strong></p>
    % <p>- na ljubljanski južni obvoznici od Rudnika proti razcepu Kozarje;
    %     <br>- na gorenjski avtocesti od Kosez do predora Šentvid proti Kranju;
    %     <br>- na štajerski avtocesti med Blagovico in Trojanami proti Celju ter pred prehodom Šentilj proti Avstriji;
    %     <br>- na ljubljanski zahodni obvoznici od Kosez proti Kozarjam ter na primorski avtocesti med Ljubljano in Brezovico proti Kopru;
    %     <br>- na cestah Ljubljana - Brezovica, Lesce - Bled in Šmarje - Koper.
    % </p>